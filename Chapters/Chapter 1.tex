\ifx\allfiles\undefined

	% 如果有这一部分另外的package,在这里加上
	% 没有的话不需要
	
	\begin{document}
\else
\fi
    \chapter{引言}
	\section{考核方式及成绩构成}
	\textcolor{red}{\textbf{本处说明在可预见的时间范围内仅适用于梁军老师班级的同学。}}
	\begin{itemize}
	\item 本课程没有期末大考;
	\item 本课程的成绩构成为:平时成绩+期末成绩,其中期末成绩占比60\%;
	\item 平时成绩包括:
	\begin{itemize}
		\item 课外练习,主要涉及运筹学的基本概念理论、尤其算法,有一些编程工作;
		\item 课堂讨论;
	\end{itemize}
	\item 期末成绩包括:
	\begin{itemize}
		\item 课程结束时有一个30分钟左右随堂小考(闭卷),主要涉及运筹学的基本概念、方法;
		\item 期末大作业含大作业的研究报告和结题答辩;
	\end{itemize}
	\end{itemize}

	\section{大作业}
	\subsection{内容与形式}
	\begin{enumerate}
		\item 大作业以小组为单位(每个小组4-6人,1位组长,人员自由组阁);要求第二周就要分好组(定下组员及组长),第三周就要初步定下题目(以后可以改),并开始调研工作;
		\item 为帮助大家定题和开展工作,第三周结束前由老师抽一点时间与大家交流一下;另外,近几年的范例会发给大家;
		\item 每个小组就自选的某个社会问题或工程问题,结合课程所学知识和方法进行分析、建模、优化、预测、评价、决策,最后给出解决问题的定性、定量化建议或解决方案,并撰写成科技报告或科技论文(格式见后);
		\item 将大作业论文整理成ppt,进行课堂现场宣讲和答辩,由任课教师、助教和其他同学作为评委和质询者,教师、助教和同学们当堂评分(同学们的分数以小组名义给出,每个小组给一个分数),所以大作业答辩课小组成员尽量坐在一起;
		\item 答辩过程中小组同学拍照留念(集体照、演讲照、回答问题交流照),代表性照片插入到研究报告中;
		\item 答辩后各小组根据情况(老师和同学们的建议、存在的问题未尽的内容等等),进行修改、定稿;
		\item 关于讨论课和大作业完成后所提交的材料:word+ppt+pdf文件是必须的,如果在准备中用到或形成了一些电子材料(如图片、视频、程序、软件、下载的资料、其他),也一并上交(显示了同学们在讨论课环节上的深入和广泛程度,展现了工作量),加分更多;
	\end{enumerate}
	\subsection{格式}
	\begin{enumerate}
		\item 题目、人员及分工、指导老师(2个老师都写)、开始时间、结束时间;
		\item 中文摘要、关键词,英文摘要、关键词;
		\item 正文;
		\begin{enumerate}
			\item 引言或问题的提出或需要解决的问题或研究目的等等-说清楚做什么;
			\item 理论方法方面的描述、讨论、引用等等-说清楚用什么方法做和怎么做的;
			\item 具体应用或解决方案或实验结果等等-做的结果如何;
			\item 分析、讨论和结论-得到了什么;
		\end{enumerate}
		\item 参考文献;
		\item 附录、附件(一些电子材料,如图片、视频、程序、软件、下载的资料、其他);
	\end{enumerate}

\ifx\allfiles\undefined
	
	% 如果有这一部分的参考文献的话,在这里加上
	% 没有的话不需要
	% 因此各个部分的参考文献可以分开放置
	% 也可以统一放在主文件末尾。
	
	%  bibfile.bib是放置参考文献的文件,可以用zotero导出。
	% \bibliography{bibfile}
	
	end{document}
	\else
	\fi