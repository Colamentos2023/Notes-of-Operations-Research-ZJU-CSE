\ifx\allfiles\undefined

    % 如果有这一部分另外的package,在这里加上
    % 没有的话不需要
    
    \begin{document}
\else
\fi
\section{第四章作业}
\subsection{车间生产问题(分枝定界法)}
\textbf{Q:}设某服装加工厂有5个生产车间,可以用6种不同的成品布料(单位为m)加工不同的服装销售。对于第$i$个生产车间分别利用第$j$种布料进行加工生产后,可以获得利润为$r_{ij}$(元/m)$(i=1,2,...,5; j=1,2,...,6)$,具体的数据表如表4.5所示。\\
该厂现有资金40万元,为了分配这些资金,根据各车间的实际生产需求,工厂要求每个车间每种布料至少加工1000米,每个车间的总加工能力最多10000米,那么试问该工厂每种布料应购买多少米,又如何分配给所属的5个车间,使得总利润最大?

\begin{table}[ht]
\centering
\begin{tabular}{|c|c|c|c|c|c|c|}
\hline
\multirow{2}{*}{布料} & \multicolumn{6}{c|}{加工利润/元} \\ \cline{2-7} 
                      & 1 & 2 & 3 & 4 & 5 & 6 \\ \hline
车间一 & 4 & 3 & 4 & 4 & 5 & 6 \\ \hline
车间二 & 3 & 4 & 5 & 3 & 4 & 5 \\ \hline
车间三 & 5 & 3 & 4 & 5 & 5 & 4 \\ \hline
车间四 & 3 & 3 & 4 & 4 & 6 & 6 \\ \hline
车间五 & 3 & 3 & 3 & 4 & 5 & 7 \\ \hline
布料单价/元/米 & 6 & 6 & 7 & 8 & 9 & 10 \\ \hline
\end{tabular}
\caption{布料单价及加工利润}
\end{table}

\textbf{A:分析如下。}\\
为最大化服装加工厂的总利润,定义以下变量和参数:

\begin{itemize}
    \item 令$I = \{1, 2, 3, 4, 5\}$为车间集合
    \item 令$J = \{1, 2, 3, 4, 5, 6\}$为布料集合
    \item 令$x_{ij}$为车间$i$分配的布料$j$的米数(整数,单位:米)
    \item 令$r_{ij}$为车间$i$使用布料$j$的单位利润(元/米),由表4.5给出
    \item 令$c_j$为布料$j$的单价(元/米),由表4.5给出
\end{itemize}

\textbf{目标函数}\\
最大化总利润:
\[
\max z = \sum_{i=1}^5 \sum_{j=1}^6 r_{ij} x_{ij}
\]

\textbf{约束条件}
\begin{enumerate}
    \item 预算约束:
    \[
    \sum_{i=1}^5 \sum_{j=1}^6 c_j x_{ij} \leq 400000
    \]
    \item 最低加工量约束:
    \[
    x_{ij} \geq 1000 \quad \forall i = 1, 2, \ldots, 5; \quad \forall j = 1, 2, \ldots, 6
    \]
    \item 车间容量约束:
    \[
    \sum_{j=1}^6 x_{ij} \leq 10000 \quad \forall i = 1, 2, \ldots, 5
    \]
    \item 整数约束:
    \[
    x_{ij} \in \mathbb{Z}^+, \quad x_{ij} \geq 1000 \quad \forall i, j
    \]
\end{enumerate}

\begin{codebox}{Matlab代码(修正版)}{1}
\begin{verbatim}
% 服装加工厂整数规划问题求解
clear; clc;

% 定义参数
m = 5; % 车间数
n = 6; % 布料种类
total_var = m * n;

% 利润矩阵和单价
R = [4 3 4 4 5 6; 3 4 5 3 4 5; 5 3 4 5 5 4; 
     3 3 4 4 6 6; 3 3 3 4 5 7];
c = [6 6 7 8 9 10];

% 目标函数
f = -R(:);

% 约束构造
A = zeros(m+1, total_var);
b = zeros(m+1, 1);

% 预算约束
A(1,:) = repelem(c, 1, m);
b(1) = 400000;

% 车间容量约束(修正关键错误)
for i = 1:m
    A(i+1, i:m:end) = 1; % 正确索引每个车间的所有布料
    b(i+1) = 10000;
end

% 变量范围
lb = 1000 * ones(total_var, 1);
ub = inf(total_var, 1);
intcon = 1:total_var;

% 求解
[x, fval] = intlinprog(f, intcon, A, b, [], [], lb, ub);

% 结果输出
x = reshape(x, [m, n]);
total_profit = -fval;
fprintf('最大总利润: %.2f元\n', total_profit);
disp('布料分配方案(x_ij, 单位:米):'); disp(x);

% 每种布料购买量
fabric_amount = sum(x, 1);
fprintf('每种布料购买量(米):\n');
for j = 1:n
    fprintf('布料%d: %.0f米\n', j, fabric_amount(j));
end
\end{verbatim}
\end{codebox}

\textbf{求解方法}\\
采用分枝定界法求解该整数规划问题。通过MATLAB的\texttt{intlinprog}函数实现,关键修正在于车间容量约束的索引方式,确保正确限制每个车间的总加工量。

\textbf{实验结果}
\begin{itemize}
    \item \textbf{最大总利润}:243,000元
    \item \textbf{布料分配方案}(单位:米):
    \[
    \begin{bmatrix}
    1000 & 1000 & 1000 & 1000 & 1000 & 5000 \\
    1000 & 1000 & 5000 & 1000 & 1000 & 1000 \\
    5000 & 1000 & 1000 & 1000 & 1000 & 1000 \\
    1000 & 1000 & 1000 & 1000 & 3000 & 3000 \\
    1000 & 5000 & 1000 & 1000 & 1000 & 1000
    \end{bmatrix}
    \]
    \item \textbf{每种布料购买量}:
    \begin{itemize}
        \item 布料1:9000米
        \item 布料2:9000米
        \item 布料3:9000米
        \item 布料4:5000米
        \item 布料5:7000米
        \item 布料6:15000米
    \end{itemize}
\end{itemize}

\textbf{结果验证}
\begin{itemize}
    \item \textbf{预算约束}:总成本$=9000\times6+9000\times6+9000\times7+5000\times8+7000\times9+15000\times10=400,000$元,满足
    \item \textbf{最低加工量}:所有$x_{ij} \geq 1000$,满足
    \item \textbf{车间容量}:各车间总量均为10,000米(如车间5:$1000+5000+1000+1000+1000+1000=10,000$),满足
    \item \textbf{利润验证}:车间4利润$=3\times1000+3\times1000+4\times1000+4\times1000+6\times3000+6\times3000=52,000$元,总利润243,000元验证通过
\end{itemize}

\subsection{旅行者背包问题(0-1规划)}
\textbf{Q:}首先列写模型,然后自己赋值进行程序求解如下题目:一个旅行者要在背包里装一些最有用的东西,但限制最多只能携带 bkg 件物品,每件物品只能是整件携带,对每件物品都规定了一定的“使用价值”(有用的程度),如果共有 n 件物品,第 j 件物品重 $a_j$ kg,其价值为 $c_j (j=1,2,…,n)$,问题是:在携带的物品总重量不超过 bkg 的条件下,携带哪些物品可使总价值最大?
\\
\textbf{A:分析如下。}
\textbf{数学模型}\\
定义以下变量和参数:
\begin{itemize}
    \item \( n \):物品总数。
    \item \( b \):背包最大承重量。
    \item \( a_j \):第 \( j \) 件物品的重量。
    \item \( c_j \):第 \( j \) 件物品的价值。
    \item \( x_j \):二元变量,表示是否携带第 \( j \) 件物品(\( x_j = 1 \) 表示携带,\( x_j = 0 \) 表示不携带)。
\end{itemize}

\textbf{目标函数}:
\[
\max z = \sum_{j=1}^n c_j x_j
\]

\textbf{约束条件}:
\begin{enumerate}
    \item 重量约束:
    \[
    \sum_{j=1}^n a_j x_j \leq b
    \]
    \item 二元约束:
    \[
    x_j \in \{0, 1\} \quad \forall j = 1, 2, \ldots, n
    \]
\end{enumerate}

\textbf{具体实例}\\
我们选择以下数据:
\begin{itemize}
    \item \( n = 5 \)(物品总数)。
    \item \( b = 10 \)(背包最大承重)。
    \item 物品的重量和价值如下:
\end{itemize}

\begin{center}
\begin{tabular}{|c|c|c|}
\hline
\( j \) & \( a_j \) (kg) & \( c_j \) \\
\hline
1 & 2 & 3 \\
2 & 3 & 4 \\
3 & 4 & 5 \\
4 & 5 & 6 \\
5 & 6 & 7 \\
\hline
\end{tabular}
\end{center}

\textbf{MATLAB代码}\\
以下是使用显枚举法求解的MATLAB代码:

\begin{codebox}{显枚举法Matlab代码}{1}
\begin{amzcode}{matlab}
% 0-1背包问题求解(显枚举法)
clear; clc;

% 参数定义
n = 5; % 物品总数
b = 10; % 背包最大承重
a = [2, 3, 4, 5, 6]; % 物品重量
c = [3, 4, 5, 6, 7]; % 物品价值

% 初始化变量
max_val = 0; % 最大价值
best_comb = []; % 最优组合

% 枚举所有可能的组合
for i = 0:2^n - 1
    comb = zeros(1,n); % 当前组合
    for j = 1:n
        comb(j) = bitget(i,j); % 使用bitget获取二进制表示
    end
    total_weight = sum(a .* comb); % 计算总重量
    if total_weight <= b % 如果总重量不超过背包容量
        total_value = sum(c .* comb); % 计算总价值
        if total_value > max_val % 更新最大价值和最优组合
            max_val = total_value;
            best_comb = comb;
        end
    end
end

% 输出结果
fprintf('最大价值: %d\n', max_val);
fprintf('选中的物品: ');
for j = 1:n
    if best_comb(j) == 1
        fprintf('%d ', j);
    end
end
fprintf('\n');
\end{amzcode}
\end{codebox}

\begin{codebox}{动态规划法Matlab代码}{1}
	\begin{amzcode}{matlab}
% 动态规划法求解
dp = zeros(n+1, b+1); % 初始化dp表
for i = 1:n
    for w = 0:b
        idx_w = w + 1; % MATLAB索引从1开始
        if a(i) > w
            dp(i+1, idx_w) = dp(i, idx_w); % 不能携带第i件物品
        else
            take = dp(i, idx_w - a(i)) + c(i); % 携带第i件物品
            not_take = dp(i, idx_w); % 不携带第i件物品
            dp(i+1, idx_w) = max(take, not_take); % 取最大值
        end
    end
end
max_val_dp = dp(n+1, b+1); % 获取最大价值
fprintf('动态规划法最大价值: %d\n', max_val_dp);
\end{amzcode}
\end{codebox}

\textbf{求解方法}\\
采用显枚举法求解该0-1背包问题。由于物品数量较少(\( n = 5 \)),共有 \( 2^5 = 32 \) 种组合,显枚举法计算复杂度可接受。求解步骤如下:
\begin{enumerate}
    \item 枚举所有可能的物品组合(使用二进制表示,从 \( 0 \) 到 \( 2^n - 1 \))。
    \item 对每种组合,计算总重量 \( \sum_{j=1}^n a_j x_j \),检查是否满足 \( \leq b \)。
    \item 若满足重量约束,计算总价值 \( \sum_{j=1}^n c_j x_j \),更新最大价值和最优组合。
    \item 输出最大价值和选中的物品。
\end{enumerate}
为验证结果,采用动态规划法,通过构建二维表格 \( dp \),计算前 \( i \) 件物品在容量 \( w \) 下的最大价值,确保结果正确。

\textbf{MATLAB代码思路}\\
MATLAB代码通过以下步骤实现:
\begin{itemize}
    \item \textbf{参数初始化}:定义物品数量(\( n = 5 \))、背包容量(\( b = 10 \))、物品重量数组 \( a \)、价值数组 \( c \)。
    \item \textbf{显枚举法}:
        \begin{itemize}
            \item 使用循环遍历 \( 0 \) 到 \( 2^n - 1 \),通过二进制位提取每种组合。
            \item 计算每种组合的总重量和总价值,筛选满足重量约束的组合。
            \item 记录最大价值和对应的物品组合。
        \end{itemize}
    \item \textbf{动态规划法}:
        \begin{itemize}
            \item 构建二维数组 \( dp \),其中 \( dp[i][w] \) 表示前 \( i \) 件物品在容量 \( w \) 下的最大价值。
            \item 使用递推公式更新 \( dp \),比较携带和不携带第 \( i \) 件物品的价值。
        \end{itemize}
    \item \textbf{求解与输出}:输出显枚举法的最优解(最大价值和选中的物品),并用动态规划法验证最大价值。
\end{itemize}

\textbf{实验结果}\\
运行MATLAB代码,得到以下结果:
\begin{itemize}
    \item \textbf{最大价值}:13
    \item \textbf{选中的物品}:物品 1, 2, 4
    \item \textbf{动态规划验证}:最大价值为 13,与显枚举法一致。
\end{itemize}

\textbf{结果验证}
\begin{itemize}
    \item \textbf{重量约束}:选中的物品 1、2、4 的总重量为 \( 2 + 3 + 5 = 10 \leq 10 \),满足约束。
    \item \textbf{价值计算}:总价值为 \( 3 + 4 + 6 = 13 \),与输出一致。
    \item \textbf{其他组合验证}:
        \begin{itemize}
            \item 物品 1、2、3:重量 \( 2 + 3 + 4 = 9 \leq 10 \),价值 \( 3 + 4 + 5 = 12 < 13 \)。
            \item 物品 3、5:重量 \( 4 + 6 = 10 \leq 10 \),价值 \( 5 + 7 = 12 < 13 \)。
            \item 物品 2、4:重量 \( 3 + 5 = 8 \leq 10 \),价值 \( 4 + 6 = 10 < 13 \)。
        \end{itemize}
    \item \textbf{动态规划验证}:动态规划法计算的最大价值为 13,与显枚举法结果一致。
\end{itemize}

\ifx\allfiles\undefined
    
    % 如果有这一部分的参考文献的话,在这里加上
    % 没有的话不需要
    % 因此各个部分的参考文献可以分开放置
    % 也可以统一放在主文件末尾。
    
    %  bibfile.bib是放置参考文献的文件,可以用zotero导出。
    % \bibliography{bibfile}
    
    end{document}
    \else
    \fi