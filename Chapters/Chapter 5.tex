\ifx\allfiles\undefined

	% 如果有这一部分另外的package,在这里加上
	% 没有的话不需要
	
	\begin{document}
\else
\fi
    \chapter{非线性规划}
    \section{非线性规划问题与数学模型}
    \subsection{非线性规划问题的定义}
    
\begin{dfnbox}{非线性规划}{amznotes}
    如果目标函数或约束条件方程中存在任何非线性因子,则问题\textbf{非线性规划}。
\end{dfnbox}
非线性规划具有以下几个特点:
\begin{itemize}
    \item 目标函数或约束条件方程中存在任何\textcolor{red}{非线性因子},或全部为\textcolor{red}{非线性函数}:
    \item \textcolor{red}{没有统一的、通用的解法}(这不同于线性规划,它有单纯形法作为通用解法),目前各种解法都有自己的适用范围;
    \item 非常难以采用解析解法,更多采用\textcolor{red}{数值解法};
    \item 非线性规划的最优解不像线性规划那样在边界点达到,而是\textcolor{red}{有可能在可行域中任一点}达到;
    \item 非线性规划存在\textcolor{red}{局部极值点}和\textcolor{red}{全局极值点}之分,这一点也与线性规划不同。
\end{itemize}
\begin{dfnbox}{局部极值点}{amznotes}
    设 $x^* \in K$,如果存在某个 $\epsilon > 0$,使得有与 $x$ 距离小于 $\epsilon$ 的 $x \in K$(即 $x \in K \cap \|x - x^*\| < \epsilon$),均满足不等式 $f(x) \geq f(x^*)$,则称 $x^*$ 为 $f(x)$ 在 $K$ 上的局部极小点,$f(x^*)$ 为局部极小值。\\
    若 $x \neq x^*$ 且 $f(x) > f(x^*)$,则 $x^*$ 为严格局部极小点,$f(x^*)$ 为严格局部极小值。
\end{dfnbox}
\begin{dfnbox}{全局极值点}{amznotes}
    设 $x^* \in K$,如果对于任意 $x \in K$,都有 $f(x) \geq f(x^*)$,则称 $x^*$ 为 $f(x)$ 在 $K$ 上的全局极小点,$f(x^*)$ 为全局极小值。\\
    若 $x \neq x^*$ 且 $f(x) > f(x^*)$,则 $x^*$ 为严格全局极小点,$f(x^*)$ 为严格全局极小值。
\end{dfnbox}
两点说明:
\begin{itemize}
    \item 局部极值点和全局极值点在定义上的主要区别是点的比较范围是x*附近的一个小邻域,还是整个K域。
    \item 局部极小点可能同时是全局极小点,全局极小点同时一定是局部极小点。
\end{itemize}
\begin{figure}[H]
    \centering
    \includegraphics[width=0.5\textwidth]{./image/19.png}
    \caption{非线性规划也有局部极值和全局极值的区分}
    \label{fig:Chapter4_Temporary_Pavilion_1}
\end{figure}
\begin{figure}[H]
    \centering
    \begin{subfigure}{0.42\textwidth}
        \centering
        \includegraphics[width=\linewidth]{image/17.png}
        \caption{线性规划中,当目标函数取定值时,其等值几何图形为直线或平面。}
    \end{subfigure}
    \hfill
    \begin{subfigure}{0.4\textwidth}
        \centering
        \includegraphics[width=\linewidth]{image/18.png}
        \caption{非线性规划中,等值线为规则或不规则的曲线、曲面,类似于地图上的等高线}
    \end{subfigure}
    \caption{线性规划与非线性规划的对比}
\end{figure}

\begin{exbox}{资源分配问题}
1\textbf{例:}有 A 和 B 两种资源,数量分别为 a 和 b,用于生产 n 种产品。如果 A 种资源以数量 \(x_k\),B 种资源以数量 \(y_k\),用于生产第 k 种产品,其收益为 \(g_k(x_k, y_k)\),问如何分配这两种资源用于 n 种产品的生产使总收益最大?
\\
\textbf{解:}

由题意,以 \(x_k\) 和 \(y_k\)(\(k = 1, 2, \dots, n\))为决策变量,以生产 n 种产品的总收益为目标函数,资源的总量为约束条件,则问题的优化模型为:

\[
\max z = g_1(x_1, y_1) + g_2(x_2, y_2) + \dots + g_n(x_n, y_n)
\]

约束条件:

\[
\begin{cases}
x_1 + x_2 + \dots + x_n = a, \\
y_1 + y_2 + \dots + y_n = b, \\
x_k \geq 0, \quad y_k \geq 0 \quad (k = 1, 2, \dots, n).
\end{cases}
\]
\end{exbox}

\begin{exbox}{最小圆盘问题}
    1\textbf{例:}设平面上有 $m$ 个点,找覆盖这 $m$ 个点的最小圆盘。

    \textbf{解:} 设 $m$ 个点为 $p_i, \, i = 1, 2, \dots, m$,则平面上任一点 $x$ 到这 $m$ 个点的距离最大者满足
    
    \[
    f(x) = \max_{1 \leq i \leq m} \| x - p_i \|
    \]
    
    则以 $x$ 为圆心,$f(x)$ 为半径的圆盘必覆盖这 $m$ 个点,于是问题转化为求解最小半径的圆盘问题:
    
    \[
    \min_x \max_{1 \leq i \leq m} \| x - p_i \|
    \]
    
    这是一个无约束的非线性规划问题。
\end{exbox}
\textcolor{red}{若没有约束条件,称为无约束极值问题。否则称为约束极值问题。}
\subsection{非线性规划问题的数学模型}
\begin{thmbox}{一般形式}{cool}
    \begin{align*}
        \text{max} \quad & z = g_1(x_1, y_1) + g_2(x_2, y_2) + \cdots + g_n(x_n, y_n) \\
        \text{subject to} \quad & \sum_{k=1}^{n} x_k = a, \\
        & \sum_{k=1}^{n} y_k = b, \\
        & x_k \geq 0, \ y_k \geq 0 \quad (k = 1, 2, \dots, n).
        \end{align*}
        
        \begin{align*}
        \text{min} \quad & \max_{1 \leq l \leq m} \left\{ \| x - p_l \| \right\}
    \end{align*}
\end{thmbox}
\begin{thmbox}{标准模型I}{cool}
    \begin{align*}
        \min \quad & f(x) \\
        \text{subject to} \quad & h_i(x) = 0, \quad i = 1, 2, \dots, m \quad \text{(m 个等式约束)} \\
        & g_j(x) \geq 0, \quad j = 1, 2, \dots, l \quad \text{(l 个不等式约束)} \\
        & x \in \mathbb{R}^n
        \end{align*}
        
        \bigskip
        \text{若有另一种形式,可以转化为上述形式,例如:}
        \begin{align*}
        \max \quad  f(x) \quad &\Rightarrow \quad \min (-f(x)) \\
        g_j(x) \leq 0  \quad &\Rightarrow \quad -g_j(x) \geq 0
        \end{align*}
\end{thmbox}
\begin{thmbox}{标准模型II}{cool}
    \begin{align*}
        \min \quad & f(x) \\
        \text{subject to} \quad & g_i(x) \geq 0, \quad i = 1, 2, \dots, n \quad \text{(n 个不等式约束)} \\
        & x \in \mathbb{R}^n
    \end{align*}
\end{thmbox}


\begin{notebox}{\textbf{模型 I 与模型 II 的转换:}}
\\
主要是把等式约束化为不等式约束:
\begin{align*}
    h_i(x) = 0 \quad \Rightarrow \quad \begin{cases}
        h_i(x) \geq 0 \\
        h_i(x) \leq 0 \quad \Rightarrow \quad -h_i(x) \geq 0
    \end{cases}
    \Rightarrow
    g_i(x)\geq 0 
\end{align*}
\end{notebox}
\begin{notebox}{\textbf{求解方法}}
    \\求解非线性规划问题常有两种方法:
    \begin{enumerate}
        \item \textbf{解析法}:必要条件+充分条件;
        \item \textbf{数值法}:迭代。
    \end{enumerate}
\end{notebox}


\subsection{求解非线性规划问题的解析法}
为了使用解析方法解决非线性规划问题,我们首先引出几个定义\footnote{读者在微积分或数学分析课程中中学过以上概念。如果感到有些遗忘,可以理解为“梯度”就相当于一元函数的一阶导数,“Hesse 矩阵”就相当于一元函数的二阶导数。我们在判断一元函数的极值点的时候,就是通过求一阶导为0,二阶导大于或小于0来判断是极小值或极大值的。对于多元函数也是类似的道理。}:
\begin{dfnbox}{函数的梯度}{amznotes}
    设 $x \in K \subset \mathbb{R}^n$,$f(x)$ 在 $K$ 上有一阶连续偏导数,则
    \[
    \nabla f(x) \equiv \left[ \frac{\partial f(x)}{\partial x_1}, \frac{\partial f(x)}{\partial x_2}, \ldots, \frac{\partial f(x)}{\partial x_n} \right]^T
    \]
    称函数 $f(x)$ 在 $x$ 处的\textbf{梯度}。
\end{dfnbox}
\textbf{梯度的几何意义}:
    $\nabla f(x)$ 是 $f(x)$ 在 $x$ 处增加最快或减少最快的速度,也是 $f(x)$ 的图形在 $x$ 处的陡峭程度。
\begin{dfnbox}{平稳点}{amznotes}
    若$\nabla f(x)=0$,则称$x$是$f(x)$的\textbf{平稳点}。
\end{dfnbox}

\begin{dfnbox}{海森Hesse矩阵}{amznotes}
    设 $f(x)$ 在 $K$ 上有二阶连续偏导数,则
    \[
    H(x) \equiv \begin{bmatrix}
        \frac{\partial^2 f(x)}{\partial x_1^2} & \frac{\partial^2 f(x)}{\partial x_1 \partial x_2} & \cdots & \frac{\partial^2 f(x)}{\partial x_1 \partial x_n} \\
        \frac{\partial^2 f(x)}{\partial x_2 \partial x_1} & \frac{\partial^2 f(x)}{\partial x_2^2} & \cdots & \frac{\partial^2 f(x)}{\partial x_2 \partial x_n} \\
        \vdots & \vdots & \ddots & \vdots \\
        \frac{\partial^2 f(x)}{\partial x_n \partial x_1} & \frac{\partial^2 f(x)}{\partial x_n \partial x_2} & \cdots & \frac{\partial^2 f(x)}{\partial x_n^2}
    \end{bmatrix}
    \]
    称函数 $f(x)$ 在 $x$ 处的 \textbf{Hesse 矩阵}。
\end{dfnbox}
\begin{dfnbox}{矩阵的正定性,负定性,不定性}{amznotes}
    对于对称矩阵 $A$,若对任意非零向量 $X \neq 0$,二次型为正,即 $X^T A X > 0$,则称该二次型为正定二次型,$A$ 为正定矩阵\footnote{在线性代数中,我们学习过判断正定矩阵的方法有特征值判定法(所有特征值大于0)、主子式判定法(如果所有阶的顺序主子式,即从左上角开始的子矩阵的行列式都大于0);此外,如果两个矩阵都是正定的,那他们的和也是正定的;一个正定矩阵的逆矩阵也是正定的。}。\\
    若 $X^T A X \geq 0 \Rightarrow$ 半正定\\
    若 $X^T A X < 0 \Rightarrow$ 负定\\
    若 $X^T A X \leq 0 \Rightarrow$ 半负定\\
    若既非正定,又非负定,则为不定。
\end{dfnbox}
\begin{thmbox}{定理一:极值点必要条件}{cool}
    若$f(x)$在其存在一阶连续偏导数的区域内达到局部极值点,则该极值点必为平稳点。
\end{thmbox}
此处要注意:
\begin{itemize}
    \item 该定理是必要条件,反之不一定成立,即\textcolor{red}{平稳点不一定是极值点},例如鞍点。
    \item $f(x)$在其不满足一阶连续偏导数的区域内的极值点,不一定满足定理一。
\end{itemize}
\begin{figure}[H]
    \centering
    \begin{subfigure}{0.42\textwidth}
        \centering
        \includegraphics[width=\linewidth]{image/20.png}
        \caption{鞍点是平稳点但不是极值点}
    \end{subfigure}
    \hfill
    \begin{subfigure}{0.4\textwidth}
        \centering
        \includegraphics[width=\linewidth]{image/21.png}
        \caption{一阶偏导数不连续的极值点不一定是平稳点}
    \end{subfigure}
    \caption{定理一是必要条件的理解}
\end{figure}
为了能够正确判断极值点,我们使用定理二:
\begin{thmbox}{定理二:局部极值点判断的充要条件}{cool}
    若 $f(x)$ 在 $K$ 上具有二阶连续偏导数,并且满足:
    \begin{enumerate}
        \item $x^* \in K$ 是平稳点,即 $\nabla f(x^*) = 0$
        \item $H(x^*)$ 是正定矩阵(Hesse 矩阵)
    \end{enumerate}
    则 $x^* \in K$ 是 $f(x)$ 的严格局部极小点。
\end{thmbox}
\subsection{求解非线性规划问题的数值法}

\begin{dfnbox}{下降迭代算法}{amznotes}
    若由某算法产生的解序列 $\{ X^{(k)} \}$ 使目标函数值 $f(X^{(k)})$ 逐步减小,则称这组算法为\textbf{下降迭代算法}。
\end{dfnbox}

\begin{notebox}{\textbf{下降迭代算法}}{}
    \begin{itemize}
        \item \textbf{迭代法的基本思路:}
        \\
        初始估计 $X^{(0)}$ \quad $\xrightarrow{\text{按某种算法}}$ \quad 比 $X^{(0)}$ 更好的 $X^{(1)}$ \quad $\xrightarrow{\text{按算法}}$ \quad $X^{(2)} \cdots X^{(n)}$
        $\xrightarrow{\text{得到解序列}} \quad \{ X^{(0)}, X^{(1)}, \cdots, X^{(n)} \cdots \}$
    
        \bigskip
        若解序列收敛于 $X^*$,即,
        \[
        \lim_{k \to \infty} \| X^{(k)} - X^* \| = 0
        \]
        则称 $\{ X^{(0)}, X^{(1)}, \cdots, X^{(n)} \cdots \}$ 收敛于 $X^*$。
        \\
    \end{itemize}
\end{notebox}
几个关键问题是:
\begin{enumerate}
    \item 迭代算法的初始点 $X^{(0)}$ 的选择?
    \item 算法的设计,以当前点依据何种原则构建下一个点?
    \item 迭代算法的终止条件?
\end{enumerate}
\subsubsection{迭代法的基本步骤}
\begin{itemize}
\item \textbf{一、初值的确定}\\初值的确定没有特别的办法,很多情况下依靠经验,或求解者对问题的了解程度来确定的。初值应尽可能与可能的最优解靠得近一些。
\item \textbf{二、算法的设计}
\begin{enumerate}
    \item 选定某一初始点 $X^{(0)}$,并令 $k=0$。

    \item 确定能使 $f(X)$ 下降的搜索方向 $P^{(k)}$。

    \item 从 $X^{(k)}$ 出发行,沿方向 $P^{(k)}$ 确定迭代步长 $\lambda_k$。

    \item 确定,产生下一个迭代点 $X^{(k+1)}$,
    \[
    X^{(k+1)} = X^{(k)} + \lambda_k P^{(k)}.
    \]
    \item 检查新点$X^{k+1}$是否为极小点,或近似极小点,或满足规定的停止条件。
    \begin{itemize}
        \item 若是,则停止迭代;
        \item 否则,令 $k=k+1$,转(2)继续进行迭代。
    \end{itemize}
\end{enumerate}
\textcolor{red}{关键在于如何选取搜索方向 $P^{(k)}$ 和步长 $\lambda_k$}\footnote{各种迭代方法的差异主要就体现在这两者的选定上。}。
\\方向的确定相对困难,但很重要,避免南辕北辙。
\\步长的确定可见\hyperref[确定迭代步长的一维搜索方法]{确定迭代步长的一维搜索方法}和\hyperref[确定迭代步长的黄金分割法]{确定迭代步长的黄金分割法}。
\item \textbf{三、算法的结束条件}
\begin{itemize}
    \item 若算法优秀且收敛的极限值确为最优解,则能通过选代找到最优解。
    \item 若迭代步数有限,只能找到近似解,当满足所要求精度时,即停止迭代。
\end{itemize}
常用以下几个方法:
\begin{enumerate}
    \item \textbf{两次迭代值的绝对误差}
    \[
    \| X^{(k+1)} - X^{(k)} \| < \varepsilon_1
    \]
    \[
    | f(X^{(k+1)}) - f(X^{(k)}) | < \varepsilon_2
    \]

    \item \textbf{两次迭代值的相对误差}
    \[
    \frac{\| X^{(k+1)} - X^{(k)} \|}{\| X^{(k)} \|} < \varepsilon_1
    \]
    \[
    \frac{| f(X^{(k+1)}) - f(X^{(k)}) |}{| f(X^{(k)}) |} < \varepsilon_2
    \]

    \item \textbf{梯度条件}
    \[
    \| \nabla f(X^{(k)}) \| < \varepsilon
    \]
\end{enumerate}
\end{itemize}

\subsubsection{确定迭代步长的一维搜索方法}
\label{确定迭代步长的一维搜索方法}
\begin{notebox}{\textbf{一维、确定步长的三种方法}}
    \begin{enumerate}
        \item \textbf{恒定步长:} 步长每次不变,计算简单,但效果较差。

        \item \textbf{变步长:} 每次人工调整步长,效果较好,但实施麻烦,且需具备较多的经验。

        \item \textbf{最速下降步长:} 使沿搜索方向使目标函数值下降最多、最快,即沿射线 $X = X^{(k)} + \lambda P^{(k)}$ 求使目标函数 $f(X)$ 的极小,
        \[
        \lambda_k : \min f(X^{(k)} + \lambda P^{(k)}),
        \]
        由于这种方法是以$\lambda$为变量的元函数 $f(X^{(k)} + \lambda P^{(k)})$ 的极小点 $\lambda_k$,故称为一维搜索,这种确定的步长为\textbf{最佳步长}。
    \end{enumerate}
\end{notebox}
以下面这个草图为例,从零点出发,方向确定为横轴正方向了,如果在横轴上走出红色的步长,纵轴下降红色的部分;但是如果在横轴走出蓝色的步长,纵轴下降的部分就会更大。这就是先确定方向,沿着搜索方向使目标函数值下降最多。
\begin{figure}[H]
    \centering
    \includegraphics[width=0.35\textwidth]{./image/22.png}
    \caption{最速下降步长}
    \label{fig:Chapter4_Temporary_Pavilion_1}
\end{figure}
实际上,读者可能会想到,如果我们每次都要实时计算步长,岂不是降低了效率吗?这个问题是有道理的,请看下面的流程图:
\begin{figure}[H]
    \centering
    \includegraphics[width=0.6\textwidth]{./image/23.png}
    \caption{流程图}
    \label{fig:Chapter4_Temporary_Pavilion_1}
\end{figure}
引入\textcolor{orange}{橙色环节}之前,可能需要迭代100次,每次执行1秒;引入橙色环节之后,可能需要迭代50次,每次执行2秒。所以总的时间谁多谁少很难说了。因此我们想到,是否有办法在迭代过程中,既能带有最优步长的色彩,计算又不过于复杂?
\subsubsection{确定迭代步长的黄金分割法(0.618法)}
\label{确定迭代步长的黄金分割法}
\begin{notebox}{\textbf{黄金分割法(0.618 法)}}{}
    \\作为一种计算更简便的方法,黄金分割法在区间缩短效率上近似斐波那契法。
    \[
    x_k' = a_{k-1} + 0.382 [b_{k-1} - a_{k-1}]
    \]
    \[
    x_k'' = a_{k-1} + 0.618 [b_{k-1} - a_{k-1}]
    \]
    其中 $x_k'$ 和 $x_k''$ 分别是 $[a_{k-1}, b_{k-1}]$ 区间内的两个点,$0.382$ 和 $0.618$ 是黄金分割法的两个常数。
    可以这么理解:$x_k'$其实是$a_k$,$x_k''$其实是$b_k$,新的$\lambda_k$为$b_{k}-a_{k}$,旧的$\lambda_{k-1}$为$a_{k-1}-b_{k-1}$ 
\end{notebox}
总结一下,为了理清楚框架,我们通过本节学习了以下内容:
\begin{enumerate}
    \item \textbf{非线性规划问题的定义}
    
    \item \textbf{非线性规划问题的数学模型(模型 I 与模型 II 的转换)}
    \begin{itemize}
        \item 一般形式
        \item 标准模型 I
        \item 标准模型 II
    \end{itemize}
    
    \item \textbf{求解非线性规划问题的方法}
    \begin{itemize}
        \item \textbf{解析法}
        \begin{itemize}
            \item 前置概念:梯度、平稳点、Hesse矩阵、正定矩阵
            \item 定理一:极值点必要条件
            \item 定理二:局部极值点判断的充要条件
        \end{itemize}
        
        \item \textbf{数值法(下降迭代算法)}
        \begin{enumerate}
            \item 初值的确定
            \item 算法的设计
            \begin{enumerate}
                \item 步长的确定
                \begin{itemize}
                    \item 恒定步长
                    \item 变步长
                    \item 最速下降步长
                    \begin{enumerate}
                        \item 一维搜索
                        \item 黄金分割法
                    \end{enumerate}
                \end{itemize}
                \item 搜索方向的确定(更为重要,后面介绍)
            \end{enumerate}
            \item 算法的结束条件
        \end{enumerate}
    \end{itemize}
\end{enumerate}


\section{无约束非线性规划问题的解}
\begin{thmbox}{无约束非线性规划}{cool}
$$
\min f(X) 
$$
\end{thmbox}
接下来介绍无约束非线性规划的几个解决办法。关键要点是确定下降的方向$P^{(k)}$,步长的计算还是用之前的内容。
\subsection{梯度法(最速下降法)}

\begin{notebox}{\textbf{梯度法(最速下降法)}}{}\\
    \noindent 一般性的数值法方式为

    \[
    X^{(k+1)} = X^{(k)} + \lambda_k P^{(k)} \quad \lambda_k \geq 0
    \]

    \bigskip
    \noindent $X^{(k)}$ ——— 第 $k$ 步迭代结果

    \bigskip
    \noindent $X^{(k+1)}$ ——— 将要进行的第 $k+1$ 步迭代点

    \bigskip
    \noindent $\lambda_k$ ——— 第 $k$ 步已迭代完成的情况下,下一步的迭代步长

    \bigskip
    \noindent $P^{(k)}$ ——— 第 $k$ 步已迭代完成的情况下,下一步的搜索方向

    \bigskip
    \noindent \textbf{每次迭代的目标}:使目标函数数值每次都有所下降,即:

    \[
    f(X^{(k+1)}) < f(X^{(k)})
    \]
    且我们希望下降的最多。
\end{notebox}
我们采用的方法如下\footnote{说人话,就是每到新的一处后算出来梯度后反着下降,到下一处再算梯度,以此类推;读者可以想到这种方法虽然提升了每次下降最快的效率,但是对于全局的效率来说未必是好的——每次下降仅考虑当前点如何最快,类似于贪心算法}:
\begin{enumerate}
    \item 为此目标,考察 $f(X)$ 在 $X^{(k)}$ 点的泰勒级数(即在 $X^{(k)}$ 对 $f(X)$ 展开):
    \[
    f(X^{(k+1)}) = f(X^{(k)} + \lambda_k P^{(k)}) \quad \text{(由一维迭代公式)}
    \]
    \[
    = f(X^{(k)}) + \lambda_k \nabla f^T(X^{(k)}) P^{(k)} + o(\lambda_k),
    \]
    \[
    \therefore \text{只要满足上式中间项一项 } \lambda_k \nabla f^T(X^{(k)}) P^{(k)} < 0 \text{,目标函数就是下降的,即 }\\
     f(X^{(k+1)}) < f(X^{(k)})
    \]
    
    \item 进一步,如果希望 $f(X^{(k+1)})$ 在 $f(X^{(k)})$ 的基础上下降的最多,则合适的 $P^{(k)}$,
    
    使泰勒展开式中间项一项 $\lambda_k \nabla f^T(X^{(k)}) P^{(k)}$ 取最大负值(此时假设 $\lambda_k$ 已选定),

    为此:
    \[
    \because \lambda_k > 0 \quad \therefore \lambda_k \nabla f^T(X^{(k)}) P^{(k)} < 0 \Rightarrow \nabla f^T(X^{(k)}) P^{(k)} < 0,
    \]

    对于 $\nabla f^T(X^{(k)}) P^{(k)} < 0$ 来说,$\nabla f^T(X^{(k)})$ 和 $P^{(k)}$ 是两个向量(且前者已确定),

    由几何学原理知道,当两个矢量大小相同,方向相反,其点积取最大负值:
    \[
    \therefore \text{取 } P^{(k)} = -\nabla f^T(X^{(k)}),
    \]
    \[
    \therefore \nabla f^T(X^{(k)}) P^{(k)} = -\nabla f^T(X^{(k)}) \cdot \nabla f(X^{(k)}) = -\|\nabla f^T(X^{(k)})\|^2 < 0,
    \]

    即为负值,又数值最小,

    \[
    \therefore \text{当取 } P^{(k)} = -\nabla f^T(X^{(k)}) \text{ 时,} f(X^{(k+1)}) \text{ 比 } f(X^{(k)}) \text{ 下降的最多,}
    \]
    
    \noindent ——称为负梯度方向。
    \item 得到了搜索方向 $P^{(k)}$,再用一维搜索进一步求取最佳的步长 $\lambda_k$,即:\\
    \[
    \text{负梯度搜索方向} + \text{一维搜索步长} \quad \Rightarrow \quad \text{最速下降法}
    \]
\end{enumerate}

\begin{notebox}{\textbf{迭代求解过程:}}{}
\\
    \noindent \textbf{Step 1:} 给定初始迭代点 $X^{(0)}$ 及容许精度 $\varepsilon > 0$,

    \bigskip
    \noindent 若 $\|\nabla f^T(X^{(0)})\|^2 \leq \varepsilon$,则 $X^{(0)}$ 即为近似最优解,停止迭代,

    \bigskip
    \noindent \textbf{Step 2:} 若 $\|\nabla f^T(X^{(0)})\|^2 > \varepsilon$,求步长 $\lambda_0$,并计算 $X^{(1)} = X^{(0)} - \lambda_0 \nabla f(X^{(0)})$,

    \bigskip
    \noindent \textbf{Step 3:} 当迭代到第 $k$ 步后,

    \bigskip
    \noindent 若 $\|\nabla f^T(X^{(k)})\|^2 < \varepsilon$,则 $X^{(k)}$ 即为近似最优解,

    \bigskip
    \noindent 否则,则确定下一个迭代点 $X^{(k+1)}$ 为,
    \[
    X^{(k+1)} = X^{(k)} - \lambda_k \nabla f(X^{(k)}),
    \]

    \noindent 并通过黄金分割法求最优的 $\lambda_k$。
\end{notebox}
    

\subsection{共轭梯度法}
这一方法主要解决的是正定二次函数最小问题,用代数的方法理解,我们知道任意非线性函数都可以由泰勒展开,用多项式无穷地逼近,
可以近似到最高阶次;同理,理论上所有的非线性规划问题都可以近似成正定二次函数最小问题。接下来我们就了解共轭、正定二
次函数最小问题的概念,以及在此基础上提出的共轭方向法,指出其不足,并了解在其基础上改进而成的共轭梯度法。\\
\begin{dfnbox}{正交与共轭}{amznotes}
    \begin{enumerate}
        \item 若 $X \in R^n, Y \in R^n$,当 $X^T Y = 0$ 时,称 $X$ 与 $Y$ 正交。
        
        \item 若 $A$ 为 $n \times n$ 对称正定矩阵,当 $X^T A Y = 0$ 时,称 $X$ 和 $Y$ 关于 $A$ 共轭,或 $X$ 和 $Y$ 为 $A$ 共轭。
        
        当 $A$ 为单位矩阵时,即为 $X$ 和 $Y$ 正交,所以\textbf{正交是共轭的一种特殊情况}。
        
        \item 若 $A$ 为 $n \times n$ 对称正定矩阵,若非零向量 $P^{(0)}, P^{(1)}, \cdots, P^{(n)} \in R^n$,关于 $A$ 两两共轭,即:
        \[
        (P^{(i)})^T A (P^{(j)}) = 0, \quad i \neq j, \quad i, j = 1, 2, \cdots, n
        \]
        则称向量组为 $A$ 共轭。
    \end{enumerate}
\end{dfnbox}
\begin{thmbox}{关于 $A$ 共轭的向量组线性无关}{cool}
    关于 $A$ 共轭的向量组线性无关。
\end{thmbox}
\begin{dfnbox}{正定二次函数最小问题}{amznotes}
    无约束极值取值问题的一个特殊情况是:
    \[
    \min f(X) = \frac{1}{2} X^T A X + B^T X + C,
    \]
    其中:$A$ 为 $n \times n$ 对称正定矩阵,$B$ 为系数向量,$C$ 为常数,

    \noindent —— 称为\textbf{正定二次函数最小问题}。
\end{dfnbox}
\begin{thmbox}{共轭方向法}{cool}
    设向量 $P^{(i)}, \ i=0,1,2,\cdots,n-1$ 为 $A$ 共轭,则从任一点 $X^{(0)}$ 出发,相继以
    \[
    P^{(0)}, P^{(1)}, \cdots, P^{(n-1)}
    \]
    为搜索方向的迭代算法:

    \[
    \begin{cases}
        \min f(X^{(k)} + \lambda P^{(k)}) = f(X^{(k)} + \lambda_k P^{(k)}), \\
        X^{(k+1)} = X^{(k)} + \lambda_k P^{(k)}
    \end{cases}
    \]

    经 $n$ 次迭代后就能收敛于正定二次函数最小问题的极值解,

    \noindent —— 称为\textbf{共轭方向法}。
\end{thmbox}
一般来说,共轭方向法仅具有理论分析意义,对于实际的正定二次函数极小问题求解,共轭方向法至少有二方面问题没有解决:
\begin{enumerate}
    \item A 共轭的搜索方向向量组理论上为\[
    P^{(0)}, P^{(1)}, \cdots, P^{(n-1)}
    \]
    但实用时如何构造这n个向量,它们具有什么形式,没有说明。
    \item 尽管在理论上只需n次迭代就能命中目标,但实际的运算误差的存在,使问题的解具有不定性。
\end{enumerate}
其中,第一点是致命的,使共轭方向法不具备可操作性。为此,发展了共轭方向法的一种具体算法——\textbf{共轭梯度法}。
\\其主要特点是: 基于梯度,构造了一组可实现的共轭方向,使算法具有可操作性。共轭梯度法是一种构造性方法。

\begin{notebox}{\textbf{非正定二次函数的共轭梯度法}}{}
    \\对于更广泛的一般无约束非线性规划,
    \[
    \min f(X),
    \]

    \noindent 可以通过逼近当前迭代点附近是够小的邻域内进行泰勒级数近似的方式,将一般非线性函数展开成二次函数,而采用具有一定迭代精度的共轭梯度法。

    \[
    f(X^{(k+1)}) = f(X^{(k)} + \lambda_k P^{(k)}),
    \]

    \[
    \approx f(X^{(k)}) + \lambda_k f^T(X^{(k)}) P^{(k)} + \frac{1}{2} \lambda_k^2 (P^{(k)})^T H(X^{(k)}) P^{(k)},
    \]

    \noindent “$\approx$”右边是二次函数。
\end{notebox}
读者可以这样理解这一方法:不同于5.1.1节,每次迭代后方向都根据梯度来进行调整,共轭梯度法是先根据梯度确定了$P^{(0)}$,再用其构建与其共轭的$P^{(1)}$,根据$P^{(1)}AP^{(0)}=0$,其中$A$和$P^{(0)}$均已知,解方程即可;之后再确定$P^{(2)}$,要求其与$P^{(0)}$和$P^{(1)}$均共轭,依次类推。\\

\subsection{变尺度法}
在前述梯度法和共轭梯度法中,确定搜索方向时充分利用了目标函数的梯度信息。假若目标函数的二阶导数信息(Hesse矩阵)也存在,我们就有理由利用二阶导数信息确定搜索方向。
\begin{notebox}{\textbf{广义牛顿法}}{}
    \\

    一种可能是,采用
    
    \[
    P^{(k)} = -H(X^{(k)})^{-1} \nabla f(X^{(k)}) 
    \]
    作为搜索方向,
    
    即
    
    \[
    X^{(k+1)} = X^{(k)} + \lambda_k P^{(k)}
    \]
    
    \[
    P^{(k)} = -H(X^{(k)})^{-1} \nabla f(X^{(k)}), \quad \lambda_k = \min_\lambda f(X^{(k)} + \lambda P^{(k)})
    \]
    称为\textbf{广义牛顿法}。
    
    \[
    P^{(k)} = -H(X^{(k)})^{-1} \nabla f(X^{(k)}) 
    \]
    称为$f(X)$在$X^{(k)}$的牛顿方向。
\end{notebox}
广义牛顿法是一个相当优秀的算法,但在应用时求 Hesse 矩阵的逆很麻烦。为了不计算 Hesse 矩阵(从而不必计算 Hesse 逆),通过构造一个近似于 Hesse矩阵的逆矩阵来满足迭代要求。
称为\textbf{变尺度法}。
\\下面来讨论如何构造$H(X^{(k)})$一的近似矩阵$\overline{H^{(k)}}$。为此,提出以下要求:
\begin{enumerate}
    \item 每做一次迭代,目标函数应下降。
    \item 每一步都能用现有信息确定下一个搜索方向。
    \item 这些近似矩阵最后应收敛于解点处的 Hesse 矩阵的逆。
\end{enumerate}

\section{约束非线性规划问题(约束极值问题)}
\begin{thmbox}{约束非线性规划}{cool}
\[
\min f(x)
\]

\[
h_i(x) = 0, \quad i = 1, 2, \dots, m
\]

\[
g_j(x) \geq 0, \quad j = 1, 2, \dots, l
\]
或

\[
\min f(x)
\]

\[
g_j(x) \geq 0, \quad j = 1, 2, \dots, l
\]

\[\because
\quad h_l(x) = 0 \iff
\begin{cases}
h_l(x) \geq 0 \\
h_l(x) \leq 0
\end{cases}
\]

\[\therefore
\text{统一采用后一形式。}
\]

若采用集合形式,则为:

\[
\min f(x), \quad x \in K \subset \mathbb{R}^n
\]

\[
K = \{x \mid g_j(x) \geq 0, \quad j = 1, 2, \dots, l\}
\]
\end{thmbox}
求解约束极值问题要注意以下两点(原则):
\begin{itemize}
    \item 目标函数在每次迭代都有所下降(涉及目标函数)
    \item 解的可行性问题,即解要在可行域中(涉及约束条件)
\end{itemize}
求解约束极值问题的三种主要方法:
\begin{enumerate}
    \item 将迭代点严格局限于可行域内
    \item 约束极值问题化为无约束问题
    \item 复杂约束极值问题化为简单约束问题
\end{enumerate}
\subsection{解析法:将约束极值问题化为无约束问题}
拉格朗日乘子法是将约束极值问题化为无约束问题的一种方法。\\
\begin{notebox}\textbf{{拉格朗日乘子法}}{}
\\对于约束条件是等式的问题,例如:
$$
\min f(x)
$$
$$
s.t. g(x)=0
$$
我们可以将其转化为:
$$
\min F(x)=f(x) + \lambda g(x)\\
$$
这样就变成了无约束的极值问题。
\end{notebox}
那么对于不等式约束条件的情况,这一思想是否还适用呢?答案是肯定的。\\
前已述及,非线性规划问题一般难以用解析法求解。然而,经过多年的研究,还是得到了许多可以用于解析求解的理论成果,库恩-塔克条件就是其中最重要的一个。
\begin{dfnbox}{K-T条件}{}
设 $x^*$ 是非线性规划:

\[
\min f(x)
\]

\[
g_j(x) \geq 0, \quad j = 1, 2, \dots, l
\]
的局部极小点,$f(x)$ 和 $g_j(x)$, $j = 1, 2, \dots, l$ 在点 $x^*$ 有一阶连续偏导数,而且在 $x^*$ 处所有起作用的梯度线性无关,则存在数 $\mu_1^*, \mu_2^*, \dots, \mu_l^*$ 使得:

\[
\nabla f(x^*) - \sum_{j=1}^{l} \mu_j^* \nabla g_j(x^*) = 0
\]

\[
\mu_j^* g_j(x^*) = 0, \quad j = 1, 2, \dots, l
\]

\[
\mu_j^* \geq 0
\]

以上称为\textbf{卡恩—塔克条件(K-T 条件)},满足该条件的点称为\textbf{卡恩—塔克点},$\mu_1^*, \mu_2^*, \dots, \mu_l^*$ 称为\textbf{拉格朗日(Lagrange)乘子}。
\end{dfnbox}
对此我们有四点说明:
\begin{enumerate}
    \item K-T条件为必要条件,只要x是极值点,且起作用约束的梯度线性无关,则该条件就成立。
    \item 满足K-T 条件的并非一定是最优点。
    \item 对于凸规划\footnote{凸规划是特别简单、特别特殊的一类,但是我们常见的问题都不属于凸规划,因此不用深究},K-T条件不但是必要条件,而且是充分条件。
    \item 得到的库恩塔克点须代入到约束条件$g_j(x)\geq0,j=1,2,\cdots,l$中验证是否满足。
\end{enumerate}
\begin{exbox}{用K-T条件求解非线性规划}{}
\textbf{例:}若有
$$
\max f(x)=(x-4)^2
$$
$$
1\leq x \leq 6
$$
\textbf{解:} 变为标准形式:

\[
\min \ f(x) = -(x-4)^2
\]
约束条件为:

\[
g_1(x) = x - 1 \geq 0, \quad g_2(x) = 6 - x \geq 0
\]
对应梯度:

\[
\nabla f(x) = -2(x - 4), \quad \nabla g_1(x) = 1, \quad \nabla g_2(x) = -1
\]
设K-T点为 \( x^* \),对两个约束条件引入拉格朗日乘子 \( \mu_1, \mu_2 \),得到K-T条件:

\[
-2(x^* - 4) - \mu_1 + \mu_2 = 0
\]

\[
\mu_1 (x^* - 1) = 0
\]

\[
\mu_2 (6 - x^*) = 0
\]

\[
\mu_1 \geq 0, \ \mu_2 \geq 0
\]

解以上方程组,得到:
\begin{itemize}
    \item (1) \( \mu_1 > 0, \ \mu_2 > 0 \), 无解
    \item (2) \( \mu_1 > 0, \ \mu_2 = 0 \), \( x^* = 1, f(x^*) = 9 \)
    \item (3) \( \mu_1 = 0, \ \mu_2 = 0 \), \( x^* = 4, f(x^*) = 0 \)
    \item (4) \( \mu_1 = 0, \ \mu_2 > 0 \), \( x^* = 6, f(x^*) = 4 \)
\end{itemize}

\textbf{结论:} 三个K-T点 \( x^* = 1, \ x^* = 4, \ x^* = 6 \),经验证 \( 1 \leq x \leq 6 \),都在可行域内。最大点为 \( x^* = 1, f(x^*) = 9 \),最小点为 \( x^* = 4, f(x^*) = 0 \)。
\end{exbox}


\subsection{数值法:可行方向法}
\section{案例分析}


\ifx\allfiles\undefined
	
	% 如果有这一部分的参考文献的话,在这里加上
	% 没有的话不需要
	% 因此各个部分的参考文献可以分开放置
	% 也可以统一放在主文件末尾。
	
	%  bibfile.bib是放置参考文献的文件,可以用zotero导出。
	% \bibliography{bibfile}
	
	end{document}
	\else
	\fi