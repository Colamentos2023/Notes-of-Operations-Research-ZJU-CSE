\ifx\allfiles\undefined

    % 如果有这一部分另外的package,在这里加上
    % 没有的话不需要
    
    \begin{document}
\else
\fi
\section{第五章作业}
\subsection{广告促销问题}
\textbf{Q:}某公司的营销管理员因经营需要,正在考虑如何安排两种不想管产品的促销活动,已知在两种产品的促销水平上的决策变量要受到资源的约束,假设用 x1, x2 表示两种产品促销活动的水平,则相应的约束为$4\mathrm{x}_1+x_2\leq20$和$\mathrm{x}_1+4x_2\leq20$,随着广告促销水平的增加,广告活动的回报会减少,从而想要取得同样程度销售增加量,就必须付出更多的广告成本。为此,营销管理员经过分析发现,对于产品1,当广告促销水平为$x$时,对应的收入为
$3x_{1}-(x_{1}-1)^{2}$(百万元);而产品2相应的收入为$3x_2-(x_2-2)^{2}$(百万元)。试为该公
司确定两种产品广告促销水平的最优组合。
\\
\textbf{A:分析如下。}

为最大化两种产品的总收入,定义以下变量:

\begin{itemize}
    \item $x_1$:产品1的促销水平。
    \item $x_2$:产品2的促销水平。
\end{itemize}

\textbf{目标函数}

最大化总收入:
\[
\max z = \left[ 3x_1 - (x_1 - 1)^2 \right] + \left[ 3x_2 - (x_2 - 2)^2 \right]
\]
简化后:
\[
z = -x_1^2 + 5x_1 - x_2^2 + 7x_2 - 5
\]

\textbf{约束条件}

\begin{enumerate}
    \item 资源约束:
    \[
    \begin{cases}
    4x_1 + x_2 \leq 20 \\
    x_1 + 4x_2 \leq 20
    \end{cases}
    \]
    \item 非负约束:
    \[
    x_1 \geq 0, \quad x_2 \geq 0
    \]
\end{enumerate}

\begin{codebox}{Matlab代码}{1}
    \begin{amzcode}{matlab}
        % 非线性规划问题求解
        clear; clc;

        % 定义目标函数
        fun = @(x) -(-x(1)^2 + 5*x(1) - x(2)^2 + 7*x(2) - 5); % 负号转为最小化

        % 初始点
        x0 = [0; 0];

        % 线性约束
        A = [4, 1; 1, 4];
        b = [20; 20];

        % 上下界
        lb = [0; 0];
        ub = [Inf; Inf];

        % 求解
        options = optimoptions('fmincon', 'Display', 'iter');
        [x, fval] = fmincon(fun, x0, A, b, [], [], lb, ub, [], options);

        % 输出结果
        fprintf('最优促销水平:x1 = %.2f, x2 = %.2f\n', x(1), x(2));
        fprintf('最大总收入:%.2f 百万元\n', -fval);
    \end{amzcode}
\end{codebox}

\textbf{求解方法}

采用非线性规划方法求解。由于目标函数为严格凹函数(二次项系数负),全局最大值存在于可行域内临界点。求解步骤如下:
\begin{enumerate}
    \item 计算偏导数,求临界点:
    \[
    \frac{\partial z}{\partial x_1} = -2x_1 + 5, \quad \frac{\partial z}{\partial x_2} = -2x_2 + 7
    \]
    解得 $x_1 = 2.5$,$x_2 = 3.5$。
    \item 验证临界点是否满足约束:
    \[
    4(2.5) + 3.5 = 13.5 \leq 20, \quad 2.5 + 4(3.5) = 16.5 \leq 20
    \]
    \item 由于目标函数凹性,确认该点为全局最大。
    \item 使用MATLAB的\texttt{fmincon}函数验证,自动求解非线性规划问题。
\end{enumerate}

\textbf{MATLAB代码思路}

MATLAB代码通过以下步骤实现:
\begin{itemize}
    \item \textbf{目标函数}:定义 $z$ 的相反数(因\texttt{fmincon}最小化)。
    \item \textbf{约束条件}:设置线性约束矩阵 $A$ 和 $b$,以及非负下界。
    \item \textbf{求解与输出}:调用\texttt{fmincon},输出最优解和最大收入。
\end{itemize}

\textbf{实验结果}

运行MATLAB代码,得到以下结果:
\begin{itemize}
    \item \textbf{最优促销水平}:$x_1 = 2.5$,$x_2 = 3.5$。
    \item \textbf{最大总收入}:13.50百万元。
\end{itemize}

\textbf{结果验证}

\begin{itemize}
    \item \textbf{约束验证}:
    \[
    4(2.5) + 3.5 = 13.5 \leq 20, \quad 2.5 + 4(3.5) = 16.5 \leq 20
    \]
    \item \textbf{收入验证}:
    \[
    z = \left[ 3(2.5) - (2.5 - 1)^2 \right] + \left[ 3(3.5) - (3.5 - 2)^2 \right] = 5.25 + 8.25 = 13.50
    \]
    \item \textbf{边界点比较}:边界点(如(0,0)、(5,0)、(0,5)、(4,4))收入分别为-5、-5、5、11,均低于13.50,确认最优。
\end{itemize}

\subsection{最优利润问题}
\textbf{Q:}按如下题意建立优化命题。设有数量为 $x_1$ 的某种原料可用于生产两种产品 A 和 B。若以数量 $y_1$ 投入生产 A, 剩下的 $x_1 - y_1$ 投入生产 B, 则利润为 $g(y_1) + h(x_1 - y_1)$, 其中 $g, h$ 为已知函数且 $g(0) = h(0) = 0$。再设 $y_1$ 和 $x_1 - y_1$ 投入生产 A 和 B 后, 可回收再利用, 回收率分别为 $a, b \in [0, 1]$, 因此在第一阶段生产后回收总量为 $x_2 = ay_1 + b(x_1 - y_1)$, 将 $x_2$ 再投入生产 A 和 B, 然后再回收……, 这样一共生产了 n 次。希望选择 $y_1, y_2, \cdots y_n$ 使总利润最大。
\\
\textbf{A:分析如下。}

为最大化n阶段生产与回收的总利润,定义以下变量:

\begin{itemize}
    \item $y_k$:第$k$阶段分配给产品A的原料数量,$k=1,2,\dots,n$。
    \item $x_k$:第$k$阶段开始时的可用原料数量,$x_1$为已知。
\end{itemize}

\textbf{目标函数}

最大化总利润:
\[
\max z = \sum_{k=1}^n \left[ g(y_k) + h(x_k - y_k) \right]
\]
其中$g(y_k)$和$h(x_k - y_k)$为产品A和B的利润函数,且$g(0) = h(0) = 0$。

\textbf{约束条件}

\begin{enumerate}
    \item 回收约束:
    \[
    x_{k+1} = a y_k + b (x_k - y_k), \quad k=1,2,\dots,n-1
    \]
    其中$a, b \in [0, 1]$为回收率。
    \item 分配约束:
    \[
    0 \leq y_k \leq x_k, \quad k=1,2,\dots,n
    \]
\end{enumerate}

\begin{codebox}{Matlab代码}{1}
    \begin{amzcode}{matlab}
        % 多阶段利润优化问题动态规划求解
        clear; clc;

        % 参数
        n = 3; % 阶段数
        x1 = 100; % 初始原料量
        a = 0.5; % 产品A回收率
        b = 0.3; % 产品B回收率
        g = @(y) y^0.5; % 利润函数g
        h = @(y) 2*y^0.5; % 利润函数h

        % 状态空间离散化
        x_max = x1;
        x_grid = 0:0.1:x_max;
        V = zeros(length(x_grid), n+1); % 价值函数

        % 动态规划:从后向前
        for k = n:-1:1
            for i = 1:length(x_grid)
                x = x_grid(i);
                y_vals = 0:0.1:x; % 离散化y
                profits = zeros(size(y_vals));
                for j = 1:length(y_vals)
                    y = y_vals(j);
                    if k == n
                        profits(j) = g(y) + h(x - y); % 最后阶段
                    else
                        x_next = a*y + b*(x - y);
                        [~, idx] = min(abs(x_grid - x_next));
                        profits(j) = g(y) + h(x - y) + V(idx, k+1);
                    end
                end
                [V(i, k), idx] = max(profits);
            end
        end

        % 输出结果
        [max_profit, idx] = max(V(:, 1));
        fprintf('最大总利润:%.2f\n', max_profit);
        fprintf('初始原料量:%.2f\n', x_grid(idx));
    \end{amzcode}
\end{codebox}

\textbf{求解方法}

采用动态规划方法求解。问题涉及多阶段决策和回收过程,动态规划通过状态变量和递推关系分解问题。求解步骤如下:
\begin{enumerate}
    \item 定义状态变量:$V_k(x)$为从第$k$阶段开始,原料量为$x$时的最大总利润。
    \item 建立递推关系:
    \[
    V_n(x) = \max_{0 \leq y \leq x} \left[ g(y) + h(x - y) \right]
    \]
    \[
    V_k(x) = \max_{0 \leq y \leq x} \left[ g(y) + h(x - y) + V_{k+1}(a y + b (x - y)) \right], \quad k=1,\dots,n-1
    \]
    \item 从$k=n$向前递推至$k=1$,计算$V_1(x_1)$。
    \item 使用MATLAB实现数值动态规划,离散化状态空间。
\end{enumerate}

\textbf{MATLAB代码思路}

MATLAB代码通过以下步骤实现:
\begin{itemize}
    \item \textbf{参数初始化}:设置$n=3$,$x_1=100$,$a=0.5$,$b=0.3$,利润函数$g(y) = y^{0.5}$,$h(y) = 2y^{0.5}$。
    \item \textbf{状态空间}:离散化原料量$x$,定义价值函数$V_k(x)$。
    \item \textbf{动态规划}:从后向前计算每阶段最大利润,考虑回收后的原料量。
    \item \textbf{求解与输出}:输出最大总利润和初始原料量。
\end{itemize}

\textbf{实验结果}

使用示例函数$g(y) = y^{0.5}$,$h(y) = 2y^{0.5}$,参数$a=0.5$,$b=0.3$,$n=3$,$x_1=100$运行,得到:
\begin{itemize}
    \item \textbf{最大总利润}:43.83。
    \item \textbf{最优分配}:通过递推追踪$y_k^*$确定。
\end{itemize}

\textbf{结果验证}

\begin{itemize}
    \item \textbf{约束验证}:确保$0 \leq y_k \leq x_k$,$x_{k+1} = 0.5 y_k + 0.3 (x_k - y_k)$满足回收约束。
    \item \textbf{合理性检查}:$g(y) = y^{0.5}$,$h(y) = 2y^{0.5}$为凹函数,动态规划保证全局最优,满足$g(0) = h(0) = 0$。
\end{itemize}

\ifx\allfiles\undefined
    
    % 如果有这一部分的参考文献的话,在这里加上
    % 没有的话不需要
    % 因此各个部分的参考文献可以分开放置
    % 也可以统一放在主文件末尾。
    
    %  bibfile.bib是放置参考文献的文件,可以用zotero导出。
    % \bibliography{bibfile}
    
    end{document}
    \else
    \fi